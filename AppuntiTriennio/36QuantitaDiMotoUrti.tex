\section{Quantità di moto e urti}
	\subsection{Quantità di moto}
	\par Si definisce quantità di moto di un oggetto il seguente vettore: \textbf{$\vec{q}=m\cdot\vec{v}$}. Ricordiamo che il prodotto di un numero positivo, come la massa, per un vettore dà come risultato un vettore il quale ha la stessa direzione e lo stesso verso del vettore di partenza e il suo modulo si ottiene moltiplicando il numero per il modulo del vettore: \textbf{$q=m\cdot v$}.
	\par Per calcolare la quantità di moto totale di più oggetti bisogna sommare le componenti delle singole quantità di moto dei singoli oggetti.
	\par\esempio calcolare la quantità di moto totale di due particelle aventi le seguenti caratteristiche:
	\par\begin{tabular}{| l | l | l |}
	\hline
	 & Particella 1 & Particella 2\\ \hline
	m & 0,500 kg & 0,500 kg\\ \hline
	v & 20,5 m/s & 32,7 m/s\\ \hline
	$\alpha$ & 50,0° & 145°\\ \hline
	\end{tabular}
	\begin{equation*}
	\vec{v_1}=
        \begin{cases}
            v_1^x=13,2 \frac{m}{s}\\
            v_1^y=15,7 \frac{m}{s}\\
        \end{cases}
        \vec{v_2}=
        \begin{cases}
            v_2^x=-26,8 \frac{m}{s}\\
            v_2^y=18,8 \frac{m}{s}\\
        \end{cases}
        \vec{q_1}=
        \begin{cases}
            q_1^x=6,60 \frac{kg\cdot m}{s}\\
            q_1^y=7,85 \frac{kg\cdot m}{s}\\
        \end{cases}
        \vec{q_2}=
        \begin{cases}
            q_2^x=-6,70 \frac{kg\cdot m}{s}\\
            q_2^y=4,70 \frac{kg\cdot m}{s}\\
        \end{cases}
    \end{equation*}
    \begin{equation*}
        \vec{q_{tot}}=
        \begin{cases}
            q_{tot}^x=-0,100 \frac{kg\cdot m}{s}\\
            q_{tot}^y=12,6 \frac{kg\cdot m}{s}\\
        \end{cases}
    \end{equation*}
    \par Se due oggetti aventi la stessa quantità di moto come modulo ma verso opposto si muovono sull'asse x (o in qualunque altra direzione), la loro quantità di moto è 0.
    \subsection{Principio di conservazione della quantità di moto}
    \par Partiamo dal 2° principio della dinamica:
    \begin{equation*}
    	\vec{a}=\frac{\vec{F_{tot}}}{m}\rightarrow
    	\frac{\Delta v}{\Delta t}=\frac{F_{tot}}{m} \rightarrow
    	\frac{\vec{v_f}-\vec{v_i}}{\Delta_t}=\frac{\vec{F_{est}}}{m}\rightarrow
    	m\cdot\vec{v_f}-m\cdot{v_i}=\vec{F_{est}}\cdot \Delta_t
    \end{equation*}
    \begin{equation} \label{eq:36variazioneQDM}
    	\vec{q_f}-\vec{q_i}=\vec{F_{est}}\cdot \Delta_t
    \end{equation}
    \par Per variare la quantità di moto (\ref{eq:36variazioneQDM}, pag. \pageref{eq:36conservazioneQDM}) è necessario applicare una forza esterna per una certa quantità di tempo.
    \subsubsection{Conservazione della quantità di moto}
    \par L'equazione \ref{eq:36conservazioneQDM} di pagina \pageref{eq:36conservazioneQDM} è un caso particolare della legge di variazione della quantità di moto:
    \begin{equation} \label{eq:36conservazioneQDM}
    	\vec{F_{est}}=0\rightarrow\vec{q_f}=\vec{q_i}
    \end{equation}
    \par\esempio un'auto ($m=1000kg$) si sta muovendo lungo l'asse x positivo con $v_i=30,0 m/s$; sapendo che la forza totale frenante abbia modulo $F_f=2500 N$, calcola il tempo necessario affinché l'auto si fermi.
    \begin{equation*}
    30000 N\cdot{s}=2500 N \cdot \Delta_t \rightarrow
    \Delta_t=\frac{30000N\cdot{s}}{2500 N}=12 s
    \end{equation*}
    \subsection{Urti}
    \par Durante un urto di qualsiasi tipo la somma di tutte le forze esterne è 0, come è 0 la somma di tutte le forze interne, quindi $q_f=q_i$ per tutti i tipi di urti.
    \par\textbf{In tutti gli urti si conserva la quantità di moto.}
    \subsubsection{Urti elastici}
    \par Si definisce urto elastico un urto nel quale, oltre alla quantità di moto, si conserva anche l'energia cinetica. Consideriamo il caso particolare di un urto elastico monodimensionale, ad esempio sull'asse x
    \par da finire