\section{Il potenziale elettrico}
\subsection{Energia potenziale elettrica}
    \par Poiché la forza elettrica e la forza gravitazionale sono matematicamente simili, possiamo dedurre il concetto di energia potenziale elettrica sfruttando le conoscenze che abbiamo sull'energia potenziale gravitazionale
    \begin{equation*}
        \vec{F_e}=q\cdot E
    \end{equation*}
    \begin{equation}\label{eq:46lavoroForzaElettrica}
        \vec{L_e}=\vec{F_e}\times \vec{s}=(q\cdot E)\cdot s \cdot \cos 0=(q\cdot E)\cdot (h_i-h_f) \cdot \cos 0=qEh_i-qEh_f
    \end{equation}
    \begin{equation}\label{eq:46epForzaElettrica}
        \vec{E_{p_e}}=\pm qEh
    \end{equation}
    \par L'equazione \ref{eq:46lavoroForzaElettrica} rappresenta il \textbf{lavoro} compiuto dalla forza elettrica, mentre l'equazione \ref{eq:46epForzaElettrica} rappresenta l'\textbf{energia potenziale elettrica}. In quest'ultima va usato il segno $-$ nel caso in cui la piastra sia carica positivamente, altrimenti va usato il segno $+$.
    \begin{equation} \label{eq:46epPiuCariche}
        E_{p_e}=\frac{kq_{1}q_2}{(d_{12})} \enspace E_{p_e}\rightarrow 0 \enspace d \rightarrow +\infty
    \end{equation}
    \par L'equazione \ref{eq:46epPiuCariche} rappresenta l'energia potenziale elettrica di un \textbf{sistema costituito da due cariche puntiformi}.
    \par In un sistema di più cariche l'$E_{p_e}$ è data dalla somma delle $E_{p_e}$ fra le coppie. con $n$ cariche puntiformi si hanno $n\choose2$ $E_{p_e}$ singole.
    \begin{equation}\label{eq:46relDeltaEpe}
        L_e=E_{p_e}^i-E_{p_e}^f
    \end{equation}
    \par Nell'equazione \ref{eq:46relDeltaEpe} viene espressa la relazione fra il lavoro compiuto dalla forza elettrica e la variazione di energia potenziale elettrica

\subsection{Il potenziale elettrico}
    \par Si definisce potenziale elettrico in un punto P dello spazio l'energia potenziale che avrebbe in quel punto la carica unitaria positiva.
    \begin{equation} \label{eq:46potenzialeElettrico}
        V=\frac{E_{p_e}}{q} \enspace [V]=\frac{J}{C}=V
    \end{equation}
    \esempio %controllare l'esempio
    \par Calcolare il potenziale elettrico al variare della distanza rispetto ad una carica $Q=5,0\cdot10^{-7}C$
    \par \textbf{Tabella}: \par \begin{tabular}{| l | l |}
        \hline
        $d$ & $V$\\ \hline
        0,1 & 44950\\ \hline
        0,5 & 8990\\ \hline
        1 & 4495\\ \hline
        1,5 & 2997\\ \hline
        2 & 2248\\ \hline
        2,5 & 1798\\ \hline
        3 & 1498\\ \hline
        3,5 & 1284\\ \hline
        4 & 1124\\ \hline
    \end{tabular} Ricordiamo che $V=\frac{kQ}{d}$
    \par \textbf{Grafico}:\par % GNUPLOT: LaTeX picture
\setlength{\unitlength}{0.240900pt}
\ifx\plotpoint\undefined\newsavebox{\plotpoint}\fi
\sbox{\plotpoint}{\rule[-0.200pt]{0.400pt}{0.400pt}}%
\begin{picture}(1500,900)(0,0)
\sbox{\plotpoint}{\rule[-0.200pt]{0.400pt}{0.400pt}}%
\put(174.0,131.0){\rule[-0.200pt]{4.818pt}{0.400pt}}
\put(112,131){\makebox(0,0){$0$}}
\put(1419.0,131.0){\rule[-0.200pt]{4.818pt}{0.400pt}}
\put(174.0,212.0){\rule[-0.200pt]{4.818pt}{0.400pt}}
\put(112,212){\makebox(0,0){$5000$}}
\put(1419.0,212.0){\rule[-0.200pt]{4.818pt}{0.400pt}}
\put(174.0,293.0){\rule[-0.200pt]{4.818pt}{0.400pt}}
\put(112,293){\makebox(0,0){$10000$}}
\put(1419.0,293.0){\rule[-0.200pt]{4.818pt}{0.400pt}}
\put(174.0,374.0){\rule[-0.200pt]{4.818pt}{0.400pt}}
\put(112,374){\makebox(0,0){$15000$}}
\put(1419.0,374.0){\rule[-0.200pt]{4.818pt}{0.400pt}}
\put(174.0,455.0){\rule[-0.200pt]{4.818pt}{0.400pt}}
\put(112,455){\makebox(0,0){$20000$}}
\put(1419.0,455.0){\rule[-0.200pt]{4.818pt}{0.400pt}}
\put(174.0,535.0){\rule[-0.200pt]{4.818pt}{0.400pt}}
\put(112,535){\makebox(0,0){$25000$}}
\put(1419.0,535.0){\rule[-0.200pt]{4.818pt}{0.400pt}}
\put(174.0,616.0){\rule[-0.200pt]{4.818pt}{0.400pt}}
\put(112,616){\makebox(0,0){$30000$}}
\put(1419.0,616.0){\rule[-0.200pt]{4.818pt}{0.400pt}}
\put(174.0,697.0){\rule[-0.200pt]{4.818pt}{0.400pt}}
\put(112,697){\makebox(0,0){$35000$}}
\put(1419.0,697.0){\rule[-0.200pt]{4.818pt}{0.400pt}}
\put(174.0,778.0){\rule[-0.200pt]{4.818pt}{0.400pt}}
\put(112,778){\makebox(0,0){$40000$}}
\put(1419.0,778.0){\rule[-0.200pt]{4.818pt}{0.400pt}}
\put(174.0,859.0){\rule[-0.200pt]{4.818pt}{0.400pt}}
\put(112,859){\makebox(0,0){$45000$}}
\put(1419.0,859.0){\rule[-0.200pt]{4.818pt}{0.400pt}}
\put(174.0,131.0){\rule[-0.200pt]{0.400pt}{4.818pt}}
\put(174,90){\makebox(0,0){$0$}}
\put(174.0,839.0){\rule[-0.200pt]{0.400pt}{4.818pt}}
\put(332.0,131.0){\rule[-0.200pt]{0.400pt}{4.818pt}}
\put(332,90){\makebox(0,0){$0.5$}}
\put(332.0,839.0){\rule[-0.200pt]{0.400pt}{4.818pt}}
\put(490.0,131.0){\rule[-0.200pt]{0.400pt}{4.818pt}}
\put(490,90){\makebox(0,0){$1$}}
\put(490.0,839.0){\rule[-0.200pt]{0.400pt}{4.818pt}}
\put(648.0,131.0){\rule[-0.200pt]{0.400pt}{4.818pt}}
\put(648,90){\makebox(0,0){$1.5$}}
\put(648.0,839.0){\rule[-0.200pt]{0.400pt}{4.818pt}}
\put(807.0,131.0){\rule[-0.200pt]{0.400pt}{4.818pt}}
\put(807,90){\makebox(0,0){$2$}}
\put(807.0,839.0){\rule[-0.200pt]{0.400pt}{4.818pt}}
\put(965.0,131.0){\rule[-0.200pt]{0.400pt}{4.818pt}}
\put(965,90){\makebox(0,0){$2.5$}}
\put(965.0,839.0){\rule[-0.200pt]{0.400pt}{4.818pt}}
\put(1123.0,131.0){\rule[-0.200pt]{0.400pt}{4.818pt}}
\put(1123,90){\makebox(0,0){$3$}}
\put(1123.0,839.0){\rule[-0.200pt]{0.400pt}{4.818pt}}
\put(1281.0,131.0){\rule[-0.200pt]{0.400pt}{4.818pt}}
\put(1281,90){\makebox(0,0){$3.5$}}
\put(1281.0,839.0){\rule[-0.200pt]{0.400pt}{4.818pt}}
\put(1439.0,131.0){\rule[-0.200pt]{0.400pt}{4.818pt}}
\put(1439,90){\makebox(0,0){$4$}}
\put(1439.0,839.0){\rule[-0.200pt]{0.400pt}{4.818pt}}
\put(174.0,131.0){\rule[-0.200pt]{0.400pt}{175.375pt}}
\put(174.0,131.0){\rule[-0.200pt]{304.738pt}{0.400pt}}
\put(1439.0,131.0){\rule[-0.200pt]{0.400pt}{175.375pt}}
\put(174.0,859.0){\rule[-0.200pt]{304.738pt}{0.400pt}}
\put(30,495){\makebox(0,0){$V$}}
\put(806,29){\makebox(0,0){$d(m)$}}
\put(174,858){\usebox{\plotpoint}}
\put(173.67,813){\rule{0.400pt}{10.840pt}}
\multiput(173.17,835.50)(1.000,-22.500){2}{\rule{0.400pt}{5.420pt}}
\put(175.17,770){\rule{0.400pt}{8.700pt}}
\multiput(174.17,794.94)(2.000,-24.943){2}{\rule{0.400pt}{4.350pt}}
\multiput(177.60,752.98)(0.468,-5.745){5}{\rule{0.113pt}{4.100pt}}
\multiput(176.17,761.49)(4.000,-31.490){2}{\rule{0.400pt}{2.050pt}}
\multiput(181.59,716.97)(0.477,-4.161){7}{\rule{0.115pt}{3.140pt}}
\multiput(180.17,723.48)(5.000,-31.483){2}{\rule{0.400pt}{1.570pt}}
\multiput(186.59,683.05)(0.485,-2.705){11}{\rule{0.117pt}{2.157pt}}
\multiput(185.17,687.52)(7.000,-31.523){2}{\rule{0.400pt}{1.079pt}}
\multiput(193.59,647.76)(0.485,-2.476){11}{\rule{0.117pt}{1.986pt}}
\multiput(192.17,651.88)(7.000,-28.879){2}{\rule{0.400pt}{0.993pt}}
\multiput(200.59,616.15)(0.488,-2.013){13}{\rule{0.117pt}{1.650pt}}
\multiput(199.17,619.58)(8.000,-27.575){2}{\rule{0.400pt}{0.825pt}}
\multiput(208.59,585.57)(0.488,-1.880){13}{\rule{0.117pt}{1.550pt}}
\multiput(207.17,588.78)(8.000,-25.783){2}{\rule{0.400pt}{0.775pt}}
\multiput(216.58,557.94)(0.491,-1.433){17}{\rule{0.118pt}{1.220pt}}
\multiput(215.17,560.47)(10.000,-25.468){2}{\rule{0.400pt}{0.610pt}}
\multiput(226.58,530.43)(0.491,-1.277){17}{\rule{0.118pt}{1.100pt}}
\multiput(225.17,532.72)(10.000,-22.717){2}{\rule{0.400pt}{0.550pt}}
\multiput(236.58,505.60)(0.491,-1.225){17}{\rule{0.118pt}{1.060pt}}
\multiput(235.17,507.80)(10.000,-21.800){2}{\rule{0.400pt}{0.530pt}}
\multiput(246.58,482.11)(0.492,-1.062){19}{\rule{0.118pt}{0.936pt}}
\multiput(245.17,484.06)(11.000,-21.057){2}{\rule{0.400pt}{0.468pt}}
\multiput(257.58,459.41)(0.492,-0.967){19}{\rule{0.118pt}{0.864pt}}
\multiput(256.17,461.21)(11.000,-19.207){2}{\rule{0.400pt}{0.432pt}}
\multiput(268.58,438.72)(0.492,-0.873){19}{\rule{0.118pt}{0.791pt}}
\multiput(267.17,440.36)(11.000,-17.358){2}{\rule{0.400pt}{0.395pt}}
\multiput(279.58,420.09)(0.492,-0.755){21}{\rule{0.119pt}{0.700pt}}
\multiput(278.17,421.55)(12.000,-16.547){2}{\rule{0.400pt}{0.350pt}}
\multiput(291.58,402.09)(0.492,-0.755){21}{\rule{0.119pt}{0.700pt}}
\multiput(290.17,403.55)(12.000,-16.547){2}{\rule{0.400pt}{0.350pt}}
\multiput(303.58,384.37)(0.492,-0.669){21}{\rule{0.119pt}{0.633pt}}
\multiput(302.17,385.69)(12.000,-14.685){2}{\rule{0.400pt}{0.317pt}}
\multiput(315.58,368.65)(0.492,-0.582){21}{\rule{0.119pt}{0.567pt}}
\multiput(314.17,369.82)(12.000,-12.824){2}{\rule{0.400pt}{0.283pt}}
\multiput(327.58,354.65)(0.492,-0.582){21}{\rule{0.119pt}{0.567pt}}
\multiput(326.17,355.82)(12.000,-12.824){2}{\rule{0.400pt}{0.283pt}}
\multiput(339.00,341.92)(0.497,-0.493){23}{\rule{0.500pt}{0.119pt}}
\multiput(339.00,342.17)(11.962,-13.000){2}{\rule{0.250pt}{0.400pt}}
\multiput(352.00,328.92)(0.496,-0.492){21}{\rule{0.500pt}{0.119pt}}
\multiput(352.00,329.17)(10.962,-12.000){2}{\rule{0.250pt}{0.400pt}}
\multiput(364.00,316.92)(0.539,-0.492){21}{\rule{0.533pt}{0.119pt}}
\multiput(364.00,317.17)(11.893,-12.000){2}{\rule{0.267pt}{0.400pt}}
\multiput(377.00,304.92)(0.600,-0.491){17}{\rule{0.580pt}{0.118pt}}
\multiput(377.00,305.17)(10.796,-10.000){2}{\rule{0.290pt}{0.400pt}}
\multiput(389.00,294.92)(0.652,-0.491){17}{\rule{0.620pt}{0.118pt}}
\multiput(389.00,295.17)(11.713,-10.000){2}{\rule{0.310pt}{0.400pt}}
\multiput(402.00,284.93)(0.728,-0.489){15}{\rule{0.678pt}{0.118pt}}
\multiput(402.00,285.17)(11.593,-9.000){2}{\rule{0.339pt}{0.400pt}}
\multiput(415.00,275.93)(0.758,-0.488){13}{\rule{0.700pt}{0.117pt}}
\multiput(415.00,276.17)(10.547,-8.000){2}{\rule{0.350pt}{0.400pt}}
\multiput(427.00,267.93)(0.824,-0.488){13}{\rule{0.750pt}{0.117pt}}
\multiput(427.00,268.17)(11.443,-8.000){2}{\rule{0.375pt}{0.400pt}}
\multiput(440.00,259.93)(0.824,-0.488){13}{\rule{0.750pt}{0.117pt}}
\multiput(440.00,260.17)(11.443,-8.000){2}{\rule{0.375pt}{0.400pt}}
\multiput(453.00,251.93)(1.123,-0.482){9}{\rule{0.967pt}{0.116pt}}
\multiput(453.00,252.17)(10.994,-6.000){2}{\rule{0.483pt}{0.400pt}}
\multiput(466.00,245.93)(0.874,-0.485){11}{\rule{0.786pt}{0.117pt}}
\multiput(466.00,246.17)(10.369,-7.000){2}{\rule{0.393pt}{0.400pt}}
\multiput(478.00,238.93)(1.123,-0.482){9}{\rule{0.967pt}{0.116pt}}
\multiput(478.00,239.17)(10.994,-6.000){2}{\rule{0.483pt}{0.400pt}}
\multiput(491.00,232.93)(1.378,-0.477){7}{\rule{1.140pt}{0.115pt}}
\multiput(491.00,233.17)(10.634,-5.000){2}{\rule{0.570pt}{0.400pt}}
\multiput(504.00,227.93)(1.378,-0.477){7}{\rule{1.140pt}{0.115pt}}
\multiput(504.00,228.17)(10.634,-5.000){2}{\rule{0.570pt}{0.400pt}}
\multiput(517.00,222.93)(1.267,-0.477){7}{\rule{1.060pt}{0.115pt}}
\multiput(517.00,223.17)(9.800,-5.000){2}{\rule{0.530pt}{0.400pt}}
\multiput(529.00,217.94)(1.797,-0.468){5}{\rule{1.400pt}{0.113pt}}
\multiput(529.00,218.17)(10.094,-4.000){2}{\rule{0.700pt}{0.400pt}}
\multiput(542.00,213.94)(1.797,-0.468){5}{\rule{1.400pt}{0.113pt}}
\multiput(542.00,214.17)(10.094,-4.000){2}{\rule{0.700pt}{0.400pt}}
\multiput(555.00,209.94)(1.797,-0.468){5}{\rule{1.400pt}{0.113pt}}
\multiput(555.00,210.17)(10.094,-4.000){2}{\rule{0.700pt}{0.400pt}}
\multiput(568.00,205.94)(1.651,-0.468){5}{\rule{1.300pt}{0.113pt}}
\multiput(568.00,206.17)(9.302,-4.000){2}{\rule{0.650pt}{0.400pt}}
\multiput(580.00,201.95)(2.695,-0.447){3}{\rule{1.833pt}{0.108pt}}
\multiput(580.00,202.17)(9.195,-3.000){2}{\rule{0.917pt}{0.400pt}}
\multiput(593.00,198.95)(2.695,-0.447){3}{\rule{1.833pt}{0.108pt}}
\multiput(593.00,199.17)(9.195,-3.000){2}{\rule{0.917pt}{0.400pt}}
\multiput(606.00,195.95)(2.695,-0.447){3}{\rule{1.833pt}{0.108pt}}
\multiput(606.00,196.17)(9.195,-3.000){2}{\rule{0.917pt}{0.400pt}}
\put(619,192.17){\rule{2.700pt}{0.400pt}}
\multiput(619.00,193.17)(7.396,-2.000){2}{\rule{1.350pt}{0.400pt}}
\multiput(632.00,190.95)(2.472,-0.447){3}{\rule{1.700pt}{0.108pt}}
\multiput(632.00,191.17)(8.472,-3.000){2}{\rule{0.850pt}{0.400pt}}
\put(644,187.17){\rule{2.700pt}{0.400pt}}
\multiput(644.00,188.17)(7.396,-2.000){2}{\rule{1.350pt}{0.400pt}}
\put(657,185.17){\rule{2.700pt}{0.400pt}}
\multiput(657.00,186.17)(7.396,-2.000){2}{\rule{1.350pt}{0.400pt}}
\put(670,183.17){\rule{2.700pt}{0.400pt}}
\multiput(670.00,184.17)(7.396,-2.000){2}{\rule{1.350pt}{0.400pt}}
\put(683,181.17){\rule{2.500pt}{0.400pt}}
\multiput(683.00,182.17)(6.811,-2.000){2}{\rule{1.250pt}{0.400pt}}
\put(695,179.17){\rule{2.700pt}{0.400pt}}
\multiput(695.00,180.17)(7.396,-2.000){2}{\rule{1.350pt}{0.400pt}}
\put(708,177.17){\rule{2.700pt}{0.400pt}}
\multiput(708.00,178.17)(7.396,-2.000){2}{\rule{1.350pt}{0.400pt}}
\put(721,175.67){\rule{3.132pt}{0.400pt}}
\multiput(721.00,176.17)(6.500,-1.000){2}{\rule{1.566pt}{0.400pt}}
\put(734,174.17){\rule{2.700pt}{0.400pt}}
\multiput(734.00,175.17)(7.396,-2.000){2}{\rule{1.350pt}{0.400pt}}
\put(747,172.67){\rule{2.891pt}{0.400pt}}
\multiput(747.00,173.17)(6.000,-1.000){2}{\rule{1.445pt}{0.400pt}}
\put(759,171.67){\rule{3.132pt}{0.400pt}}
\multiput(759.00,172.17)(6.500,-1.000){2}{\rule{1.566pt}{0.400pt}}
\put(772,170.67){\rule{3.132pt}{0.400pt}}
\multiput(772.00,171.17)(6.500,-1.000){2}{\rule{1.566pt}{0.400pt}}
\put(785,169.67){\rule{3.132pt}{0.400pt}}
\multiput(785.00,170.17)(6.500,-1.000){2}{\rule{1.566pt}{0.400pt}}
\put(798,168.67){\rule{2.891pt}{0.400pt}}
\multiput(798.00,169.17)(6.000,-1.000){2}{\rule{1.445pt}{0.400pt}}
\put(810,167.67){\rule{3.132pt}{0.400pt}}
\multiput(810.00,168.17)(6.500,-1.000){2}{\rule{1.566pt}{0.400pt}}
\put(823,166.67){\rule{3.132pt}{0.400pt}}
\multiput(823.00,167.17)(6.500,-1.000){2}{\rule{1.566pt}{0.400pt}}
\put(836,165.67){\rule{3.132pt}{0.400pt}}
\multiput(836.00,166.17)(6.500,-1.000){2}{\rule{1.566pt}{0.400pt}}
\put(849,164.67){\rule{3.132pt}{0.400pt}}
\multiput(849.00,165.17)(6.500,-1.000){2}{\rule{1.566pt}{0.400pt}}
\put(862,163.67){\rule{2.891pt}{0.400pt}}
\multiput(862.00,164.17)(6.000,-1.000){2}{\rule{1.445pt}{0.400pt}}
\put(874,162.67){\rule{3.132pt}{0.400pt}}
\multiput(874.00,163.17)(6.500,-1.000){2}{\rule{1.566pt}{0.400pt}}
\put(900,161.67){\rule{3.132pt}{0.400pt}}
\multiput(900.00,162.17)(6.500,-1.000){2}{\rule{1.566pt}{0.400pt}}
\put(913,160.67){\rule{3.132pt}{0.400pt}}
\multiput(913.00,161.17)(6.500,-1.000){2}{\rule{1.566pt}{0.400pt}}
\put(887.0,163.0){\rule[-0.200pt]{3.132pt}{0.400pt}}
\put(938,159.67){\rule{3.132pt}{0.400pt}}
\multiput(938.00,160.17)(6.500,-1.000){2}{\rule{1.566pt}{0.400pt}}
\put(926.0,161.0){\rule[-0.200pt]{2.891pt}{0.400pt}}
\put(964,158.67){\rule{3.132pt}{0.400pt}}
\multiput(964.00,159.17)(6.500,-1.000){2}{\rule{1.566pt}{0.400pt}}
\put(951.0,160.0){\rule[-0.200pt]{3.132pt}{0.400pt}}
\put(989,157.67){\rule{3.132pt}{0.400pt}}
\multiput(989.00,158.17)(6.500,-1.000){2}{\rule{1.566pt}{0.400pt}}
\put(977.0,159.0){\rule[-0.200pt]{2.891pt}{0.400pt}}
\put(1015,156.67){\rule{3.132pt}{0.400pt}}
\multiput(1015.00,157.17)(6.500,-1.000){2}{\rule{1.566pt}{0.400pt}}
\put(1002.0,158.0){\rule[-0.200pt]{3.132pt}{0.400pt}}
\put(1041,155.67){\rule{3.132pt}{0.400pt}}
\multiput(1041.00,156.17)(6.500,-1.000){2}{\rule{1.566pt}{0.400pt}}
\put(1028.0,157.0){\rule[-0.200pt]{3.132pt}{0.400pt}}
\put(1067,154.67){\rule{3.132pt}{0.400pt}}
\multiput(1067.00,155.17)(6.500,-1.000){2}{\rule{1.566pt}{0.400pt}}
\put(1054.0,156.0){\rule[-0.200pt]{3.132pt}{0.400pt}}
\put(1106,153.67){\rule{3.132pt}{0.400pt}}
\multiput(1106.00,154.17)(6.500,-1.000){2}{\rule{1.566pt}{0.400pt}}
\put(1080.0,155.0){\rule[-0.200pt]{6.263pt}{0.400pt}}
\put(1145,152.67){\rule{3.373pt}{0.400pt}}
\multiput(1145.00,153.17)(7.000,-1.000){2}{\rule{1.686pt}{0.400pt}}
\put(1119.0,154.0){\rule[-0.200pt]{6.263pt}{0.400pt}}
\put(1187,151.67){\rule{3.373pt}{0.400pt}}
\multiput(1187.00,152.17)(7.000,-1.000){2}{\rule{1.686pt}{0.400pt}}
\put(1159.0,153.0){\rule[-0.200pt]{6.745pt}{0.400pt}}
\put(1245,150.67){\rule{3.854pt}{0.400pt}}
\multiput(1245.00,151.17)(8.000,-1.000){2}{\rule{1.927pt}{0.400pt}}
\put(1201.0,152.0){\rule[-0.200pt]{10.600pt}{0.400pt}}
\put(1311,149.67){\rule{4.577pt}{0.400pt}}
\multiput(1311.00,150.17)(9.500,-1.000){2}{\rule{2.289pt}{0.400pt}}
\put(1261.0,151.0){\rule[-0.200pt]{12.045pt}{0.400pt}}
\put(1391,148.67){\rule{5.541pt}{0.400pt}}
\multiput(1391.00,149.17)(11.500,-1.000){2}{\rule{2.770pt}{0.400pt}}
\put(1330.0,150.0){\rule[-0.200pt]{14.695pt}{0.400pt}}
\put(1414.0,149.0){\rule[-0.200pt]{6.022pt}{0.400pt}}
\put(174.0,131.0){\rule[-0.200pt]{0.400pt}{175.375pt}}
\put(174.0,131.0){\rule[-0.200pt]{304.738pt}{0.400pt}}
\put(1439.0,131.0){\rule[-0.200pt]{0.400pt}{175.375pt}}
\put(174.0,859.0){\rule[-0.200pt]{304.738pt}{0.400pt}}
\end{picture}

    \subsubsection{Tabella riepilogativa dei potenziali elettrici}
    \begin{tabular}{| l | c | c |}
        \hline
        & $E_{p_e}$ & $V$\\ \hline
        Lastra piana negativa & $q \cdot E \cdot h$ & $E \cdot h$\\ \hline
        Lastra piana positiva & $-q \cdot E \cdot h$ & $-E \cdot h$\\ \hline
        Carica puntiforme & $\frac{kQq}{d}$ & $\frac{kQ}{d}$\\ \hline
    \end{tabular}
    \par In una zona dove è presente una qualche distribuzione di carica, che genera quindi un potenziale elettrico, il moto spontaneo della carica unitaria positiva avviene in direzione dei potenziali decrescenti.
    \par Si definisce \textbf{superficie equipotenziale} una superficie in cui tutti i punti hanno lo stesso potenziale.
    \subsubsection{Energia potenziale di una carica qualsiasi}
    \begin{equation*}
        E_{p_e}=V\cdot Q
    \end{equation*}
    \subsubsection{Energia potenziale di un insieme di cariche}
    \par È sufficiente sommare \textbf{algebricamente} i potenziali dovuti all'insieme di cariche
    \subsubsection{Concetto di elettronvolt}
    $1 eV$ corrisponde all'energia cinetica che acquista un elettrone quando, lasciato libero di muoversi, viene sottoposto alla differenza di potenziale di $1 V$
    \begin{equation*}
        1 eV=1,60\cdot10^{-19}J
    \end{equation*}
\subsection{Relazione fra campo elettrico e variazione del potenziale elettrico}
    \par Possiamo calcolare il lavoro compiuto dalla forza elettrica in due modi:
    \begin{equation*}
        L_e=\vec{F_e}\times \vec{s}
    \end{equation*}
    \begin{equation*}
        E_{p_e}^i=E_{p_e}^f
    \end{equation*}
    \par Perciò:
    \begin{equation*}
        q\vec{E}\times \vec{s} = q(V_i-V_f)\enspace [J]
    \end{equation*}
    \begin{equation} \label{eq:46relCampoElettricoVarEPE}
        \vec{E} \times \vec{s} = V_i-V_f\enspace [\frac{J}{C}]
    \end{equation}
    \par Il campo elettrico generato da una certa distribuzione di carica è sempre diretto dalle zone a potenziale più alto a quelle a potenziale più basso.
    \subsubsection{Circuitazione del campo elettrico}
        \par Supponiamo di avere una linea chiusa $l$ di forma qualsiasi. Dividiamo la linea $l$ in tanti pezzettini $\overrightarrow{\Delta l_k}$ ognuno dei quali è rappresentato da un vettore orientato nel verso della linea. I pezzettini $\overrightarrow{\Delta l_k}$ devono essere sufficientemente piccoli in modo tale da poter considerare ciascun trattino coincidente con una porzione di linea.
        \par Consideriamo un generico vettore $\vec{v}$ avente una direzione qualsiasi che nei vari punti della linea $l$ può avere diverse inclinazioni e diversi moduli.
        \par Si definisce \textbf{circuitazione del vettore $\vec{v}$ lungo la linea $l$} la seguente somma di prodotti scalari:
        \begin{equation}
            \Gamma_{(l)}(\vec{v})=\sum_{k=1}^{n} \vec{v}_k \times \overrightarrow{\Delta l}_k
        \end{equation}
        \par A noi interessa calcolare la circuitazione del vettore campo elettrostatico attraverso una linea continua chiusa.
        \par Supponiamo che il campo elettrico sia generato da una certa distribuzione di carica positiva nello spazio. Nei vari punti della linea $l$ il campo elettrico generato da questa distribuzione di carica avrà diverse inclinazioni e diversi valori.
        \par Per calcolare la circuitazione di $\vec{E}$ lungo la linea chiusa $l$ calcoliamo il lavoro compiuto dalla forza elettrica quando una generica carica elettrica $q$ (ad es. positiva) compie l'intero percorso.
        \begin{equation*}
            L_{tot}=\sum_{k=1}^n L_k=\sum_{k=1}^n q\vec{E}_k \times \overrightarrow{\Delta l}_k =q\sum_{k=1}^n \vec{E}_k \times \overrightarrow{\Delta l}_k
        \end{equation*}
        \par Possiamo ricordare l'ultima relazione vista fra campo elettrico e variazione del potenziale elettrico nello spazio (equazione \ref{eq:46relCampoElettricoVarEPE} pag. \pageref{eq:46relCampoElettricoVarEPE})
        \begin{equation}\label{eq:46Circuitazione}
            L_{tot}=q\sum_{k=1}^n V_n-V_{n+1} = 0 \enspace \enspace V_0=V_n
        \end{equation}
        \par La conclusione è la seguente: Il lavoro compiuto dal campo elettrico quando la carica unitaria positiva compie un qualsiasi percorso chiuso è 0, ovviamente il risultato rimane 0 anche se $q\neq1$. Abbiamo dimostrato che \textbf{la forza elettrica è conservativa}.
    \subsubsection{Distribuzione della carica nei conduttori in equilibrio elettrostatico}
        \par Ricordiamo la definizione di equilibrio:
        \begin{equation*}
            \sum_{n=0}^k \vec{F}_k=0
        \end{equation*}
        Immaginiamo una carica generica in un conduttore. Se tale carica è in equilibrio avremo
        \begin{equation*}
            \sum_{n=0}^k \vec{F}_k=0
        \end{equation*}
        \par In questa situazione consideriamo solo la forza elettrica, quindi
        \begin{equation*}
            q\cdot E_{tot}=0
        \end{equation*}
        \par Da questo deduciamo che $E_{tot}=0$ all'interno del conduttore. Questa conclusine indica che il campo elettrico all'interno del conduttore deve essere nullo ovunque.
    \subsubsection{Ubicazione delle cariche in un conduttore}
        \par Applichiamo il teorema di Gauss utilizzando una superficie appena all'interno del conduttore
        \begin{equation*}
            \Phi(\vec{E})=\frac{q_{int}}{\varepsilon_0}
        \end{equation*}
        \par Poiché $E=0$, $q_{int}=0$. Conclusione: la carica, non potendo stare all'interno, si distribuisce sulla superficie.
    \subsubsection{Campo elettrico sulla superficie di un conduttore}
        Il campo elettrico sulla superficie è perpendicolare alla stessa e vale $E=\frac{\sigma}{\varepsilon_0}$
    \subsubsection{Andamento del potenziale in un conduttore in equilibrio}
        \par Un conduttore in equilibrio è equipotenziale
\subsection{La capacità elettrica}
    \par Si definisce capacità elettrica di un conduttore il rapporto costante fra la carica presente su di esso e il potenziale assoluto in conseguenza alla presenza di tale carica.
    \begin{equation*}
        V=\frac{k\cdot Q}{R}
    \end{equation*}
    \begin{equation*}
        V=\frac{1}{4\pi\varepsilon_0}\cdot\frac{Q}{R}
    \end{equation*}
    \begin{equation}\label{eq:46Capacita}
        C=\frac{Q}{V}=4\pi\varepsilon_0\cdot R=COST
    \end{equation}
    \par Di particolare interesse è la capacità di un condensatore, che viene definita nel seguente modo: $C=\frac{Q}{\Delta V_c}$, dove con $Q$ si intende il valore assoluto della carica in una delle due armature e $\Delta V_c$ è la differenza di potenziale fra le due armature.
    \subsubsection{Caso particolare del condensatore piano}
        \begin{equation*}
            E=\frac{\mid\sigma\mid}{\varepsilon_0}
        \end{equation*}
        \begin{equation*}
            \mid\sigma\mid=\frac{\mid Q \mid}{A} \textnormal{, A è l'area}
        \end{equation*}
        \begin{equation*}
            E=\frac{\mid Q \mid}{A\varepsilon_0}
        \end{equation*}
        \par Per calcolare la $\Delta V_c$ sfruttiamo la relazione fra campo elettrico e differenza di potenziale (equazione \ref{eq:46relCampoElettricoVarEPE} pag.  \pageref{eq:46relCampoElettricoVarEPE}). Poniamo $d$ la distanza fra le armature.
        \begin{equation*}
            E\cdot d=\Delta V_c \rightarrow E=\frac{\Delta V_c}{d} \rightarrow C=\frac{E\cdot A \cdot\varepsilon_0}{E\cdot d}
        \end{equation*}
        \begin{equation}
            C=\frac{\varepsilon_0\cdot A}{d} \enspace [C]=\frac{C}{V}=F
        \end{equation}
\subsection{Composizione in serie e in parallelo di condensatori}
    \par Due o più condensatori si dicono collegati in parallelo quando sono sottoposti alla stessa variazione di potenziale.
    \par Due o più condensatori si dicono collegati in serie quando sulle aramture corrispondenti hanno la stessa carica.
    \par Ricaviamo la capacità equivalente di più condensatori in parallelo:
    \begin{equation*}
        C_{tot}=\sum_{n=0}^k C_n
    \end{equation*}
    Si chiama condensatore equivalente ($C_e$) un condensatore che, sottoposto alla stessa differenza di potenziale del sistema reale avrà una carica uguale a quella del sistema totale
\subsection{Energia associata al campo elettrico}
        \subsubsection{Significato geometrico di lavoro elettrico}
            \par Consideriamo il processo di carica di un condensatore mediante l'utilizzo di un generatore. Durante il processo di carica viene aumentata l'energia potenziale elettrica del condensatore. Da scarico ($Q=0$, $\Delta V_c=0$) l'energia potenziale elettrica era uguale a 0 e non vi era nemmeno vampo elettrico fra le aramture. Al termine del processo di carica vi sarà una $\Delta V_c$ uguale a quella del generatore, e quindi un'energia potenziale diversa da 0.
            \par Ricordiamo che la capacità di un condensatore è definita come  il rapporto costante fra la carica presente su di esso e il potenziale assoluto in conseguenza alla presenza di tale carica. La $\Delta V_c$ è direttamente proporzionale alla carica presente su ciascuna armatura.
            %grafico
            \par Ricordiamo che in varie situazioni, come ad esempio nel caso della forza elastica (paragrafo \ref{par:33forzaElastica}), e nel caso del lavoro compiuto da un gas (\large{\textbf{Da referenziare}})
