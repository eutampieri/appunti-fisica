\section{Teoria della relatività ristretta}
    \subsection{Sistemi di riferimento inerziali e principio di relatività di Galileo}
    \par Si definisce sistema di riferimento inerziale un sistema nel quale un corpo non subisce forze per effetto del moto del sistema stesso
    \par In buona approssimazione possiamo considerare la Terra come se fosse un sistema di riferimento inerziale; tutti i sistemi di riferimento in moto relativo uniforme rispetto alla Terra, di conseguenza, sono sistemi di riferimento inerziali.
    \par Convenzionalmente consideriamo due sistemi di riferimento inerziali $Oxy$ e $O'x'y'$ con gli assi $x$ e $x'$ paralleli e le origini coincidenti a $t=0$. $O'$ si muove a $\vec{V}_rel$ rispetto a $O$ e $\vec{u}$ è la velocità di un corpo rispetto ad $O$, mentre $\vec{u}'$ è la velocità rispetto a $O'$.
    \par Scriviamo le trasformazioni di Galileo per la posizione, la velocità e l'accelerazione di un corpo frs un sistema di riferimento e un altro.
    \subsubsection{Equazioni per la posizione}
    \begin{equation}
        \begin{cases}
        x = x' + V_{rel}^x(t)\\
        y = y'\\
        z = z'
        \end{cases}
    \end{equation}
    \subsubsection{Equazioni per la velocità}
    \begin{equation}
        \begin{cases}
        v_x = v_x' + V_{rel}^x\\
        v_y = v_y'\\
        v_z = v_z'
        \end{cases}
    \end{equation}
    \subsubsection{Equazioni per l'accelerazione}
    \begin{equation}
        \begin{cases}
        a_x = a_x'\\
        a_y = a_y'\\
        a_z = a_z'
        \end{cases}
    \end{equation}
    \par \textbf{Principio di relatività di Galileo}: le leggi della meccanica sono uguali in tutti i sistemi di riferimento inerziali.
    \subsubsection{Osservazioni sulle trasformazioni di Galielo}
    \par Il tempo non viene preso in considerazione nelle trasformazioni poiché si dava per scontato che osservatori inerziali diversi misurassero la distanza temporale fra due eventi in maniera identica.
    \par Quest'approccio, benché intuitivo, non è scientificamente corretto in quanto manca la verifica sperimentale