\section{Sviluppo storico della meccanica quantistica}
    \subsection{Il problema del corpo nero}
        \par Si definisce corpo nero un corpo che assorbe tutta la radiazione elettromagnetica che lo colpisce. I corpi, in funzione della loro temperatura, emettono onde elettromagnetiche secondo una legge universale chiamata legge di Stefan-Boltzmann:
        \begin{equation}
            \dot{q} = e \cdot \sigma \cdot A \cdot T^4
        \end{equation}
        \par $\dot{q}$ è l'energia emessa ogni secondo dalla superficie $A$ del corpo che si trova alla temperatura assoluta $T$ ($[\dot{q}] = W$)
        \par $\sigma$ è una costante e vale $5,67\cdot10^{-8}$ ed $e$ è il coefficiente di emissione del corpo ($0<e\leq1$).
        \par Si può dimostrare che un corpo è un buon emettitore è anche un buon assorbitore; i processi di emissione e di assorbimento di onde elettromagnetiche sono indipendenti.
        \par Per quanto riguarda l'assorbimento di onde elettromagnetiche la legge è simile:
        \begin{equation}
            \dot{q}=a\cdot\sigma\cdot A\cdot T^4
        \end{equation}
        \par Se $a=1$ il corpo viene chiamato corpo nero. Si può dimostrare che $e=a$.
        \par Si chiama potere emissivo di un corpo su tutto lo spettro l'energia emessa dal corpo nell'unità di tempo dall'unità di superficie e si indica con $I$. Se il corpo è un corpo nero, $I=\sigma\cdot T^4$.
        \par Per studiare sperimentalmente la radiazione emessa da un corpo nero si usa la cosiddetta cavità isoterma, costituita da una cavità preparata all'interno di un blocco di metallo. Si lascia un forellino da una parte. La superficie interna di questa cavità simula molto bene quella di un corpo nero.
        \par Con questo artificio si evita la fuoriuscita della radiazione riflessa e così si può vedere soltanto la radiazione caratteristica del corpo in funzione della sua temperatura.
        \par Gli studiosi hanno potuto determinare l'andamento della radiazione dei corpi (in particolare del corpo nero) al variare della temperatura. Quello che viene normalmente rappresentato nel diagramma è il potere emissivo specifico ($R(\lambda,T)$), che rappresenta l'energia irradiata del corpo nel'unità di tempo e di superficie nell'intervallo unitario di lunghezza d'onda e quindi i valori che leggeremo in ordinata si riferiscono ad una porzione unitaria di spettro.
