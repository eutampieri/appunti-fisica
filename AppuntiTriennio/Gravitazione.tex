\section{La gravitazione}
	\subsection{Le Leggi di Keplero}
		\begin{enumerate}
			\item I pianeti orbitano intorno al Sole seguendo una traiettoria ellittica; il Sole occupa uno dei fuochi dell'ellisse.
			\item Il raggio che congiunge con il Sole (chiamato raggio vettore) descrive aree uguali in tempi uguali. Questo significa che quando il pianeta è più lontano dal Sole, si muove con velocità minore.
			\item I cubi dei raggi medi delle orbite dei pianeti (oppure i cubi dei semiassi maggiori) sono direttamente proporzionali ai quadrati dei periodi di rivoluzione.
			$\dfrac{r^3}{T^2}=\dfrac{G \cdot M_{Sole}}{4\pi^2}$
		\end{enumerate}
		\subsection{Legge di gravitazione universale}
			\par La forza con cui due masse qualsiasi ($m_1$ e $m_2$) si attraggono è direttamente proporzionale al prodotto delle masse stesse e inversamente proporzionale al quadrato delle loro distanze.
			\begin{math}
			F_{12}=F_{21}=\dfrac{G \cdot m_1 \cdot m_2}{(d_{12})^2}
			\end{math}
			\par La costante \textit{G} è la \textit{costante di gravitazione universale} e vale $6,67 \cdot 10^{-11}$
			\par Le due forze sono dirette lungo la retta congiungente i centri e sono sempre attrattive (uguali e contrarie)
			\par \esempio rappresentiamo graficamente come varia la forza gravitazionale con cui la Terra attrae un satellite avente massa $m_2=200\textrm{kg}$ al variare della distanza del satellite
			\begin{center}
				\par\begin{tabular}{| l | l |}
\hline
$d_{12} \cdot 10^6$ & $F_{12} (N) \cdot 10^3$\\ \hline
6.378 & 19.60\\ \hline
7.378 & 14.64\\ \hline
8.378 & 11.36\\ \hline
9.378 & 9.065\\ \hline
10.378 & 7.402\\ \hline
11.378 & 6.158\\ \hline
\end{tabular}
				\par% GNUPLOT: LaTeX picture
\setlength{\unitlength}{0.240900pt}
\ifx\plotpoint\undefined\newsavebox{\plotpoint}\fi
\sbox{\plotpoint}{\rule[-0.200pt]{0.400pt}{0.400pt}}%
\begin{picture}(1500,900)(0,0)
\sbox{\plotpoint}{\rule[-0.200pt]{0.400pt}{0.400pt}}%
\put(174.0,131.0){\rule[-0.200pt]{4.818pt}{0.400pt}}
\put(112,131){\makebox(0,0){$6$}}
\put(1419.0,131.0){\rule[-0.200pt]{4.818pt}{0.400pt}}
\put(174.0,235.0){\rule[-0.200pt]{4.818pt}{0.400pt}}
\put(112,235){\makebox(0,0){$8$}}
\put(1419.0,235.0){\rule[-0.200pt]{4.818pt}{0.400pt}}
\put(174.0,339.0){\rule[-0.200pt]{4.818pt}{0.400pt}}
\put(112,339){\makebox(0,0){$10$}}
\put(1419.0,339.0){\rule[-0.200pt]{4.818pt}{0.400pt}}
\put(174.0,443.0){\rule[-0.200pt]{4.818pt}{0.400pt}}
\put(112,443){\makebox(0,0){$12$}}
\put(1419.0,443.0){\rule[-0.200pt]{4.818pt}{0.400pt}}
\put(174.0,547.0){\rule[-0.200pt]{4.818pt}{0.400pt}}
\put(112,547){\makebox(0,0){$14$}}
\put(1419.0,547.0){\rule[-0.200pt]{4.818pt}{0.400pt}}
\put(174.0,651.0){\rule[-0.200pt]{4.818pt}{0.400pt}}
\put(112,651){\makebox(0,0){$16$}}
\put(1419.0,651.0){\rule[-0.200pt]{4.818pt}{0.400pt}}
\put(174.0,755.0){\rule[-0.200pt]{4.818pt}{0.400pt}}
\put(112,755){\makebox(0,0){$18$}}
\put(1419.0,755.0){\rule[-0.200pt]{4.818pt}{0.400pt}}
\put(174.0,859.0){\rule[-0.200pt]{4.818pt}{0.400pt}}
\put(112,859){\makebox(0,0){$20$}}
\put(1419.0,859.0){\rule[-0.200pt]{4.818pt}{0.400pt}}
\put(174.0,131.0){\rule[-0.200pt]{0.400pt}{4.818pt}}
\put(174,90){\makebox(0,0){$6$}}
\put(174.0,839.0){\rule[-0.200pt]{0.400pt}{4.818pt}}
\put(385.0,131.0){\rule[-0.200pt]{0.400pt}{4.818pt}}
\put(385,90){\makebox(0,0){$7$}}
\put(385.0,839.0){\rule[-0.200pt]{0.400pt}{4.818pt}}
\put(596.0,131.0){\rule[-0.200pt]{0.400pt}{4.818pt}}
\put(596,90){\makebox(0,0){$8$}}
\put(596.0,839.0){\rule[-0.200pt]{0.400pt}{4.818pt}}
\put(807.0,131.0){\rule[-0.200pt]{0.400pt}{4.818pt}}
\put(807,90){\makebox(0,0){$9$}}
\put(807.0,839.0){\rule[-0.200pt]{0.400pt}{4.818pt}}
\put(1017.0,131.0){\rule[-0.200pt]{0.400pt}{4.818pt}}
\put(1017,90){\makebox(0,0){$10$}}
\put(1017.0,839.0){\rule[-0.200pt]{0.400pt}{4.818pt}}
\put(1228.0,131.0){\rule[-0.200pt]{0.400pt}{4.818pt}}
\put(1228,90){\makebox(0,0){$11$}}
\put(1228.0,839.0){\rule[-0.200pt]{0.400pt}{4.818pt}}
\put(1439.0,131.0){\rule[-0.200pt]{0.400pt}{4.818pt}}
\put(1439,90){\makebox(0,0){$12$}}
\put(1439.0,839.0){\rule[-0.200pt]{0.400pt}{4.818pt}}
\put(174.0,131.0){\rule[-0.200pt]{0.400pt}{175.375pt}}
\put(174.0,131.0){\rule[-0.200pt]{304.738pt}{0.400pt}}
\put(1439.0,131.0){\rule[-0.200pt]{0.400pt}{175.375pt}}
\put(174.0,859.0){\rule[-0.200pt]{304.738pt}{0.400pt}}
\put(30,495){\makebox(0,0){$F_{12} (N) \cdot 10^3$}}
\put(806,29){\makebox(0,0){$d_{12} \cdot 10^6$}}
\put(254,838){\usebox{\plotpoint}}
\multiput(254.58,835.26)(0.491,-0.704){17}{\rule{0.118pt}{0.660pt}}
\multiput(253.17,836.63)(10.000,-12.630){2}{\rule{0.400pt}{0.330pt}}
\multiput(264.58,821.47)(0.492,-0.637){19}{\rule{0.118pt}{0.609pt}}
\multiput(263.17,822.74)(11.000,-12.736){2}{\rule{0.400pt}{0.305pt}}
\multiput(275.58,807.47)(0.492,-0.637){19}{\rule{0.118pt}{0.609pt}}
\multiput(274.17,808.74)(11.000,-12.736){2}{\rule{0.400pt}{0.305pt}}
\multiput(286.58,793.26)(0.491,-0.704){17}{\rule{0.118pt}{0.660pt}}
\multiput(285.17,794.63)(10.000,-12.630){2}{\rule{0.400pt}{0.330pt}}
\multiput(296.58,779.47)(0.492,-0.637){19}{\rule{0.118pt}{0.609pt}}
\multiput(295.17,780.74)(11.000,-12.736){2}{\rule{0.400pt}{0.305pt}}
\multiput(307.58,765.47)(0.492,-0.637){19}{\rule{0.118pt}{0.609pt}}
\multiput(306.17,766.74)(11.000,-12.736){2}{\rule{0.400pt}{0.305pt}}
\multiput(318.58,751.43)(0.491,-0.652){17}{\rule{0.118pt}{0.620pt}}
\multiput(317.17,752.71)(10.000,-11.713){2}{\rule{0.400pt}{0.310pt}}
\multiput(328.58,738.47)(0.492,-0.637){19}{\rule{0.118pt}{0.609pt}}
\multiput(327.17,739.74)(11.000,-12.736){2}{\rule{0.400pt}{0.305pt}}
\multiput(339.58,724.62)(0.492,-0.590){19}{\rule{0.118pt}{0.573pt}}
\multiput(338.17,725.81)(11.000,-11.811){2}{\rule{0.400pt}{0.286pt}}
\multiput(350.58,711.26)(0.491,-0.704){17}{\rule{0.118pt}{0.660pt}}
\multiput(349.17,712.63)(10.000,-12.630){2}{\rule{0.400pt}{0.330pt}}
\multiput(360.58,697.62)(0.492,-0.590){19}{\rule{0.118pt}{0.573pt}}
\multiput(359.17,698.81)(11.000,-11.811){2}{\rule{0.400pt}{0.286pt}}
\multiput(371.58,684.43)(0.491,-0.652){17}{\rule{0.118pt}{0.620pt}}
\multiput(370.17,685.71)(10.000,-11.713){2}{\rule{0.400pt}{0.310pt}}
\multiput(381.58,671.62)(0.492,-0.590){19}{\rule{0.118pt}{0.573pt}}
\multiput(380.17,672.81)(11.000,-11.811){2}{\rule{0.400pt}{0.286pt}}
\multiput(392.58,658.77)(0.492,-0.543){19}{\rule{0.118pt}{0.536pt}}
\multiput(391.17,659.89)(11.000,-10.887){2}{\rule{0.400pt}{0.268pt}}
\multiput(403.58,646.43)(0.491,-0.652){17}{\rule{0.118pt}{0.620pt}}
\multiput(402.17,647.71)(10.000,-11.713){2}{\rule{0.400pt}{0.310pt}}
\multiput(413.58,633.77)(0.492,-0.543){19}{\rule{0.118pt}{0.536pt}}
\multiput(412.17,634.89)(11.000,-10.887){2}{\rule{0.400pt}{0.268pt}}
\multiput(424.58,621.77)(0.492,-0.543){19}{\rule{0.118pt}{0.536pt}}
\multiput(423.17,622.89)(11.000,-10.887){2}{\rule{0.400pt}{0.268pt}}
\multiput(435.58,609.76)(0.491,-0.547){17}{\rule{0.118pt}{0.540pt}}
\multiput(434.17,610.88)(10.000,-9.879){2}{\rule{0.400pt}{0.270pt}}
\multiput(445.58,598.77)(0.492,-0.543){19}{\rule{0.118pt}{0.536pt}}
\multiput(444.17,599.89)(11.000,-10.887){2}{\rule{0.400pt}{0.268pt}}
\multiput(456.00,587.92)(0.496,-0.492){19}{\rule{0.500pt}{0.118pt}}
\multiput(456.00,588.17)(9.962,-11.000){2}{\rule{0.250pt}{0.400pt}}
\multiput(467.58,575.76)(0.491,-0.547){17}{\rule{0.118pt}{0.540pt}}
\multiput(466.17,576.88)(10.000,-9.879){2}{\rule{0.400pt}{0.270pt}}
\multiput(477.00,565.92)(0.547,-0.491){17}{\rule{0.540pt}{0.118pt}}
\multiput(477.00,566.17)(9.879,-10.000){2}{\rule{0.270pt}{0.400pt}}
\multiput(488.00,555.92)(0.547,-0.491){17}{\rule{0.540pt}{0.118pt}}
\multiput(488.00,556.17)(9.879,-10.000){2}{\rule{0.270pt}{0.400pt}}
\multiput(499.00,545.92)(0.495,-0.491){17}{\rule{0.500pt}{0.118pt}}
\multiput(499.00,546.17)(8.962,-10.000){2}{\rule{0.250pt}{0.400pt}}
\multiput(509.00,535.92)(0.547,-0.491){17}{\rule{0.540pt}{0.118pt}}
\multiput(509.00,536.17)(9.879,-10.000){2}{\rule{0.270pt}{0.400pt}}
\multiput(520.00,525.93)(0.611,-0.489){15}{\rule{0.589pt}{0.118pt}}
\multiput(520.00,526.17)(9.778,-9.000){2}{\rule{0.294pt}{0.400pt}}
\multiput(531.00,516.93)(0.553,-0.489){15}{\rule{0.544pt}{0.118pt}}
\multiput(531.00,517.17)(8.870,-9.000){2}{\rule{0.272pt}{0.400pt}}
\multiput(541.00,507.93)(0.611,-0.489){15}{\rule{0.589pt}{0.118pt}}
\multiput(541.00,508.17)(9.778,-9.000){2}{\rule{0.294pt}{0.400pt}}
\multiput(552.00,498.93)(0.553,-0.489){15}{\rule{0.544pt}{0.118pt}}
\multiput(552.00,499.17)(8.870,-9.000){2}{\rule{0.272pt}{0.400pt}}
\multiput(562.00,489.93)(0.692,-0.488){13}{\rule{0.650pt}{0.117pt}}
\multiput(562.00,490.17)(9.651,-8.000){2}{\rule{0.325pt}{0.400pt}}
\multiput(573.00,481.93)(0.611,-0.489){15}{\rule{0.589pt}{0.118pt}}
\multiput(573.00,482.17)(9.778,-9.000){2}{\rule{0.294pt}{0.400pt}}
\multiput(584.00,472.93)(0.626,-0.488){13}{\rule{0.600pt}{0.117pt}}
\multiput(584.00,473.17)(8.755,-8.000){2}{\rule{0.300pt}{0.400pt}}
\multiput(594.00,464.93)(0.692,-0.488){13}{\rule{0.650pt}{0.117pt}}
\multiput(594.00,465.17)(9.651,-8.000){2}{\rule{0.325pt}{0.400pt}}
\multiput(605.00,456.93)(0.692,-0.488){13}{\rule{0.650pt}{0.117pt}}
\multiput(605.00,457.17)(9.651,-8.000){2}{\rule{0.325pt}{0.400pt}}
\multiput(616.00,448.93)(0.721,-0.485){11}{\rule{0.671pt}{0.117pt}}
\multiput(616.00,449.17)(8.606,-7.000){2}{\rule{0.336pt}{0.400pt}}
\multiput(626.00,441.93)(0.692,-0.488){13}{\rule{0.650pt}{0.117pt}}
\multiput(626.00,442.17)(9.651,-8.000){2}{\rule{0.325pt}{0.400pt}}
\multiput(637.00,433.93)(0.798,-0.485){11}{\rule{0.729pt}{0.117pt}}
\multiput(637.00,434.17)(9.488,-7.000){2}{\rule{0.364pt}{0.400pt}}
\multiput(648.00,426.93)(0.721,-0.485){11}{\rule{0.671pt}{0.117pt}}
\multiput(648.00,427.17)(8.606,-7.000){2}{\rule{0.336pt}{0.400pt}}
\multiput(658.00,419.93)(0.798,-0.485){11}{\rule{0.729pt}{0.117pt}}
\multiput(658.00,420.17)(9.488,-7.000){2}{\rule{0.364pt}{0.400pt}}
\multiput(669.00,412.93)(0.798,-0.485){11}{\rule{0.729pt}{0.117pt}}
\multiput(669.00,413.17)(9.488,-7.000){2}{\rule{0.364pt}{0.400pt}}
\multiput(680.00,405.93)(0.721,-0.485){11}{\rule{0.671pt}{0.117pt}}
\multiput(680.00,406.17)(8.606,-7.000){2}{\rule{0.336pt}{0.400pt}}
\multiput(690.00,398.93)(0.798,-0.485){11}{\rule{0.729pt}{0.117pt}}
\multiput(690.00,399.17)(9.488,-7.000){2}{\rule{0.364pt}{0.400pt}}
\multiput(701.00,391.93)(0.943,-0.482){9}{\rule{0.833pt}{0.116pt}}
\multiput(701.00,392.17)(9.270,-6.000){2}{\rule{0.417pt}{0.400pt}}
\multiput(712.00,385.93)(0.721,-0.485){11}{\rule{0.671pt}{0.117pt}}
\multiput(712.00,386.17)(8.606,-7.000){2}{\rule{0.336pt}{0.400pt}}
\multiput(722.00,378.93)(0.943,-0.482){9}{\rule{0.833pt}{0.116pt}}
\multiput(722.00,379.17)(9.270,-6.000){2}{\rule{0.417pt}{0.400pt}}
\multiput(733.00,372.93)(0.798,-0.485){11}{\rule{0.729pt}{0.117pt}}
\multiput(733.00,373.17)(9.488,-7.000){2}{\rule{0.364pt}{0.400pt}}
\multiput(744.00,365.93)(0.852,-0.482){9}{\rule{0.767pt}{0.116pt}}
\multiput(744.00,366.17)(8.409,-6.000){2}{\rule{0.383pt}{0.400pt}}
\multiput(754.00,359.93)(0.943,-0.482){9}{\rule{0.833pt}{0.116pt}}
\multiput(754.00,360.17)(9.270,-6.000){2}{\rule{0.417pt}{0.400pt}}
\multiput(765.00,353.93)(0.852,-0.482){9}{\rule{0.767pt}{0.116pt}}
\multiput(765.00,354.17)(8.409,-6.000){2}{\rule{0.383pt}{0.400pt}}
\multiput(775.00,347.93)(0.943,-0.482){9}{\rule{0.833pt}{0.116pt}}
\multiput(775.00,348.17)(9.270,-6.000){2}{\rule{0.417pt}{0.400pt}}
\multiput(786.00,341.93)(0.943,-0.482){9}{\rule{0.833pt}{0.116pt}}
\multiput(786.00,342.17)(9.270,-6.000){2}{\rule{0.417pt}{0.400pt}}
\multiput(797.00,335.93)(0.852,-0.482){9}{\rule{0.767pt}{0.116pt}}
\multiput(797.00,336.17)(8.409,-6.000){2}{\rule{0.383pt}{0.400pt}}
\multiput(807.00,329.93)(0.943,-0.482){9}{\rule{0.833pt}{0.116pt}}
\multiput(807.00,330.17)(9.270,-6.000){2}{\rule{0.417pt}{0.400pt}}
\multiput(818.00,323.93)(0.943,-0.482){9}{\rule{0.833pt}{0.116pt}}
\multiput(818.00,324.17)(9.270,-6.000){2}{\rule{0.417pt}{0.400pt}}
\multiput(829.00,317.93)(1.044,-0.477){7}{\rule{0.900pt}{0.115pt}}
\multiput(829.00,318.17)(8.132,-5.000){2}{\rule{0.450pt}{0.400pt}}
\multiput(839.00,312.93)(0.943,-0.482){9}{\rule{0.833pt}{0.116pt}}
\multiput(839.00,313.17)(9.270,-6.000){2}{\rule{0.417pt}{0.400pt}}
\multiput(850.00,306.93)(1.155,-0.477){7}{\rule{0.980pt}{0.115pt}}
\multiput(850.00,307.17)(8.966,-5.000){2}{\rule{0.490pt}{0.400pt}}
\multiput(861.00,301.93)(1.044,-0.477){7}{\rule{0.900pt}{0.115pt}}
\multiput(861.00,302.17)(8.132,-5.000){2}{\rule{0.450pt}{0.400pt}}
\multiput(871.00,296.93)(0.943,-0.482){9}{\rule{0.833pt}{0.116pt}}
\multiput(871.00,297.17)(9.270,-6.000){2}{\rule{0.417pt}{0.400pt}}
\multiput(882.00,290.93)(1.155,-0.477){7}{\rule{0.980pt}{0.115pt}}
\multiput(882.00,291.17)(8.966,-5.000){2}{\rule{0.490pt}{0.400pt}}
\multiput(893.00,285.93)(1.044,-0.477){7}{\rule{0.900pt}{0.115pt}}
\multiput(893.00,286.17)(8.132,-5.000){2}{\rule{0.450pt}{0.400pt}}
\multiput(903.00,280.93)(1.155,-0.477){7}{\rule{0.980pt}{0.115pt}}
\multiput(903.00,281.17)(8.966,-5.000){2}{\rule{0.490pt}{0.400pt}}
\multiput(914.00,275.93)(1.155,-0.477){7}{\rule{0.980pt}{0.115pt}}
\multiput(914.00,276.17)(8.966,-5.000){2}{\rule{0.490pt}{0.400pt}}
\multiput(925.00,270.94)(1.358,-0.468){5}{\rule{1.100pt}{0.113pt}}
\multiput(925.00,271.17)(7.717,-4.000){2}{\rule{0.550pt}{0.400pt}}
\multiput(935.00,266.93)(1.155,-0.477){7}{\rule{0.980pt}{0.115pt}}
\multiput(935.00,267.17)(8.966,-5.000){2}{\rule{0.490pt}{0.400pt}}
\multiput(946.00,261.93)(1.044,-0.477){7}{\rule{0.900pt}{0.115pt}}
\multiput(946.00,262.17)(8.132,-5.000){2}{\rule{0.450pt}{0.400pt}}
\multiput(956.00,256.94)(1.505,-0.468){5}{\rule{1.200pt}{0.113pt}}
\multiput(956.00,257.17)(8.509,-4.000){2}{\rule{0.600pt}{0.400pt}}
\multiput(967.00,252.93)(1.155,-0.477){7}{\rule{0.980pt}{0.115pt}}
\multiput(967.00,253.17)(8.966,-5.000){2}{\rule{0.490pt}{0.400pt}}
\multiput(978.00,247.94)(1.358,-0.468){5}{\rule{1.100pt}{0.113pt}}
\multiput(978.00,248.17)(7.717,-4.000){2}{\rule{0.550pt}{0.400pt}}
\multiput(988.00,243.93)(1.155,-0.477){7}{\rule{0.980pt}{0.115pt}}
\multiput(988.00,244.17)(8.966,-5.000){2}{\rule{0.490pt}{0.400pt}}
\multiput(999.00,238.94)(1.505,-0.468){5}{\rule{1.200pt}{0.113pt}}
\multiput(999.00,239.17)(8.509,-4.000){2}{\rule{0.600pt}{0.400pt}}
\multiput(1010.00,234.94)(1.358,-0.468){5}{\rule{1.100pt}{0.113pt}}
\multiput(1010.00,235.17)(7.717,-4.000){2}{\rule{0.550pt}{0.400pt}}
\multiput(1020.00,230.94)(1.505,-0.468){5}{\rule{1.200pt}{0.113pt}}
\multiput(1020.00,231.17)(8.509,-4.000){2}{\rule{0.600pt}{0.400pt}}
\multiput(1031.00,226.94)(1.505,-0.468){5}{\rule{1.200pt}{0.113pt}}
\multiput(1031.00,227.17)(8.509,-4.000){2}{\rule{0.600pt}{0.400pt}}
\multiput(1042.00,222.94)(1.358,-0.468){5}{\rule{1.100pt}{0.113pt}}
\multiput(1042.00,223.17)(7.717,-4.000){2}{\rule{0.550pt}{0.400pt}}
\multiput(1052.00,218.94)(1.505,-0.468){5}{\rule{1.200pt}{0.113pt}}
\multiput(1052.00,219.17)(8.509,-4.000){2}{\rule{0.600pt}{0.400pt}}
\multiput(1063.00,214.94)(1.505,-0.468){5}{\rule{1.200pt}{0.113pt}}
\multiput(1063.00,215.17)(8.509,-4.000){2}{\rule{0.600pt}{0.400pt}}
\multiput(1074.00,210.94)(1.358,-0.468){5}{\rule{1.100pt}{0.113pt}}
\multiput(1074.00,211.17)(7.717,-4.000){2}{\rule{0.550pt}{0.400pt}}
\multiput(1084.00,206.95)(2.248,-0.447){3}{\rule{1.567pt}{0.108pt}}
\multiput(1084.00,207.17)(7.748,-3.000){2}{\rule{0.783pt}{0.400pt}}
\multiput(1095.00,203.94)(1.505,-0.468){5}{\rule{1.200pt}{0.113pt}}
\multiput(1095.00,204.17)(8.509,-4.000){2}{\rule{0.600pt}{0.400pt}}
\multiput(1106.00,199.94)(1.358,-0.468){5}{\rule{1.100pt}{0.113pt}}
\multiput(1106.00,200.17)(7.717,-4.000){2}{\rule{0.550pt}{0.400pt}}
\multiput(1116.00,195.95)(2.248,-0.447){3}{\rule{1.567pt}{0.108pt}}
\multiput(1116.00,196.17)(7.748,-3.000){2}{\rule{0.783pt}{0.400pt}}
\multiput(1127.00,192.94)(1.358,-0.468){5}{\rule{1.100pt}{0.113pt}}
\multiput(1127.00,193.17)(7.717,-4.000){2}{\rule{0.550pt}{0.400pt}}
\multiput(1137.00,188.95)(2.248,-0.447){3}{\rule{1.567pt}{0.108pt}}
\multiput(1137.00,189.17)(7.748,-3.000){2}{\rule{0.783pt}{0.400pt}}
\multiput(1148.00,185.95)(2.248,-0.447){3}{\rule{1.567pt}{0.108pt}}
\multiput(1148.00,186.17)(7.748,-3.000){2}{\rule{0.783pt}{0.400pt}}
\multiput(1159.00,182.94)(1.358,-0.468){5}{\rule{1.100pt}{0.113pt}}
\multiput(1159.00,183.17)(7.717,-4.000){2}{\rule{0.550pt}{0.400pt}}
\multiput(1169.00,178.95)(2.248,-0.447){3}{\rule{1.567pt}{0.108pt}}
\multiput(1169.00,179.17)(7.748,-3.000){2}{\rule{0.783pt}{0.400pt}}
\multiput(1180.00,175.95)(2.248,-0.447){3}{\rule{1.567pt}{0.108pt}}
\multiput(1180.00,176.17)(7.748,-3.000){2}{\rule{0.783pt}{0.400pt}}
\multiput(1191.00,172.94)(1.358,-0.468){5}{\rule{1.100pt}{0.113pt}}
\multiput(1191.00,173.17)(7.717,-4.000){2}{\rule{0.550pt}{0.400pt}}
\multiput(1201.00,168.95)(2.248,-0.447){3}{\rule{1.567pt}{0.108pt}}
\multiput(1201.00,169.17)(7.748,-3.000){2}{\rule{0.783pt}{0.400pt}}
\multiput(1212.00,165.95)(2.248,-0.447){3}{\rule{1.567pt}{0.108pt}}
\multiput(1212.00,166.17)(7.748,-3.000){2}{\rule{0.783pt}{0.400pt}}
\multiput(1223.00,162.95)(2.025,-0.447){3}{\rule{1.433pt}{0.108pt}}
\multiput(1223.00,163.17)(7.025,-3.000){2}{\rule{0.717pt}{0.400pt}}
\multiput(1233.00,159.95)(2.248,-0.447){3}{\rule{1.567pt}{0.108pt}}
\multiput(1233.00,160.17)(7.748,-3.000){2}{\rule{0.783pt}{0.400pt}}
\multiput(1244.00,156.95)(2.248,-0.447){3}{\rule{1.567pt}{0.108pt}}
\multiput(1244.00,157.17)(7.748,-3.000){2}{\rule{0.783pt}{0.400pt}}
\multiput(1255.00,153.95)(2.025,-0.447){3}{\rule{1.433pt}{0.108pt}}
\multiput(1255.00,154.17)(7.025,-3.000){2}{\rule{0.717pt}{0.400pt}}
\multiput(1265.00,150.94)(1.505,-0.468){5}{\rule{1.200pt}{0.113pt}}
\multiput(1265.00,151.17)(8.509,-4.000){2}{\rule{0.600pt}{0.400pt}}
\multiput(1276.00,146.95)(2.248,-0.447){3}{\rule{1.567pt}{0.108pt}}
\multiput(1276.00,147.17)(7.748,-3.000){2}{\rule{0.783pt}{0.400pt}}
\multiput(1287.00,143.95)(2.025,-0.447){3}{\rule{1.433pt}{0.108pt}}
\multiput(1287.00,144.17)(7.025,-3.000){2}{\rule{0.717pt}{0.400pt}}
\multiput(1297.00,140.95)(2.248,-0.447){3}{\rule{1.567pt}{0.108pt}}
\multiput(1297.00,141.17)(7.748,-3.000){2}{\rule{0.783pt}{0.400pt}}
\put(174.0,131.0){\rule[-0.200pt]{0.400pt}{175.375pt}}
\put(174.0,131.0){\rule[-0.200pt]{304.738pt}{0.400pt}}
\put(1439.0,131.0){\rule[-0.200pt]{0.400pt}{175.375pt}}
\put(174.0,859.0){\rule[-0.200pt]{304.738pt}{0.400pt}}
\end{picture}

			\end{center}
			\subsubsection{Sulla superficie della Terra}
				Come mai \begin{math}F_g=m\cdot g=\dfrac{G\cdot m_1 \cdot m_2}{{d_{12}}^2}\end{math}? Dobbiamo allora assumere che $g=\dfrac{G\cdot m_T}{d^2}$.
				\par Sulla superficie della Terra i tre fattori $G$, $m_T$ e $d^2$sono0 sempre costanti e valgono circa \textbf{$9,81 \dfrac{\textrm{N}}{\textrm{kg}}$}. Questa grandezza viene chiamata \textbf{campo gravitazionale (g)}. Se lo si calcola sulla superficie della Terra il suo valore medio è $9,81 \dfrac{\textrm{N}}{\textrm{kg}}$. Dando diversi valori di $d$ è possibile calcolare il campo gravitazionale
				\par \esempio calcoliamo il nostro peso ad una distanza di 30000 km dalla Terra.
				\begin{center}
					\par \begin{math}
					g=\dfrac{G\cdot m_T}{d^2}=\dfrac{6,67\cdot 10^{-11} \cdot 5.97\cdot10^{24}}{30000000^2}=0,4424433 \end{math}
					\par\begin{math}
					P=m\cdot g=70\cdot 0,4424433=30,97\textrm{N}
					\end{math}
				\end{center}
				\par Guardando la definizione matematica, è possibile dare la seguente definizione di campo gravitazionale: \textbf{$g=\dfrac{F_g}{m}$}. Essa rappresenta la forza gravitazionale che subirebbe in un certo punto P dello spazio una massa di 1kg. $g$ è anche accelerazione ($\dfrac{N}{kg}=\dfrac{m}{s^2}$).
		\subsection{Energia potenziale gravitazionale (masse puntiformi)}
			$L_g=E_{pg}^i-e_{pg}^f$
			$E_{pg}=\dfrac{G \cdot m_1 \cdot m_2}{(d_{12})^2}\longrightarrow E_{pg}=0$ con $d_{12}$ tendente all'infinito
			\par \esempio valutiamo il lavoro compiuto da $F_g$ quando un oggetto di massa $m=100 \textrm{kg}$ scende da una quota $h_i=10000 \textrm{m}$ ad una quota $h_f=0 \textrm{m}$
			\begin{center}
				\par $g=\dfrac{Gm_t}{d^2}\simeq 9,77$
				\par $g_{medio}=\dfrac{9,81+9,77}{2}=9,79$
				\par $L_g=9,79\cdot 10^6 \textrm{J}$
				\par $L_g=E_{pg}^i-E_{pg}^f=-\dfrac{Gm_Tm}{d_i}+\dfrac{Gm_Tm}{d_ff}=-\dfrac{3,986\cdot10^{16}}{6,388\cdot10^6}+\dfrac{3,986\cdot10^{16}}{6,388\cdot10^6}=-6,240\cdot10^9+6,250\cdot1o^9=9,78\cdot10^6$
			\end{center}
		\subsection{Conservazione dell'energia meccanica nei fenomeni gravitazionali}
			\par$E_m^i=E_m^f \longrightarrow E_{pg}^i+E_c=E_{pg}^f+E_c^f \rightarrow-\dfrac{Gm_1m_2}{d_{12}}+\dfrac{1}{2}mv_i^2=-\dfrac{Gm_1m_2}{d_{12}}+\dfrac{1}{2}mv_f^2$
	\subsection{Formulario}				